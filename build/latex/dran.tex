%% Generated by Sphinx.
\def\sphinxdocclass{report}
\documentclass[letterpaper,10pt,english]{sphinxmanual}
\ifdefined\pdfpxdimen
   \let\sphinxpxdimen\pdfpxdimen\else\newdimen\sphinxpxdimen
\fi \sphinxpxdimen=.75bp\relax
\ifdefined\pdfimageresolution
    \pdfimageresolution= \numexpr \dimexpr1in\relax/\sphinxpxdimen\relax
\fi
%% let collapsible pdf bookmarks panel have high depth per default
\PassOptionsToPackage{bookmarksdepth=5}{hyperref}

\PassOptionsToPackage{booktabs}{sphinx}
\PassOptionsToPackage{colorrows}{sphinx}

\PassOptionsToPackage{warn}{textcomp}
\usepackage[utf8]{inputenc}
\ifdefined\DeclareUnicodeCharacter
% support both utf8 and utf8x syntaxes
  \ifdefined\DeclareUnicodeCharacterAsOptional
    \def\sphinxDUC#1{\DeclareUnicodeCharacter{"#1}}
  \else
    \let\sphinxDUC\DeclareUnicodeCharacter
  \fi
  \sphinxDUC{00A0}{\nobreakspace}
  \sphinxDUC{2500}{\sphinxunichar{2500}}
  \sphinxDUC{2502}{\sphinxunichar{2502}}
  \sphinxDUC{2514}{\sphinxunichar{2514}}
  \sphinxDUC{251C}{\sphinxunichar{251C}}
  \sphinxDUC{2572}{\textbackslash}
\fi
\usepackage{cmap}
\usepackage[T1]{fontenc}
\usepackage{amsmath,amssymb,amstext}
\usepackage{babel}



\usepackage{tgtermes}
\usepackage{tgheros}
\renewcommand{\ttdefault}{txtt}



\usepackage[Bjarne]{fncychap}
\usepackage{sphinx}

\fvset{fontsize=auto}
\usepackage{geometry}


% Include hyperref last.
\usepackage{hyperref}
% Fix anchor placement for figures with captions.
\usepackage{hypcap}% it must be loaded after hyperref.
% Set up styles of URL: it should be placed after hyperref.
\urlstyle{same}

\addto\captionsenglish{\renewcommand{\contentsname}{Contents:}}

\usepackage{sphinxmessages}
\setcounter{tocdepth}{1}



\title{dran}
\date{Aug 17, 2023}
\release{v1.0}
\author{Pfesi van Zyl}
\newcommand{\sphinxlogo}{\vbox{}}
\renewcommand{\releasename}{Release}
\makeindex
\begin{document}

\ifdefined\shorthandoff
  \ifnum\catcode`\=\string=\active\shorthandoff{=}\fi
  \ifnum\catcode`\"=\active\shorthandoff{"}\fi
\fi

\pagestyle{empty}
\sphinxmaketitle
\pagestyle{plain}
\sphinxtableofcontents
\pagestyle{normal}
\phantomsection\label{\detokenize{index::doc}}



\chapter{Introduction}
\label{\detokenize{index:introduction}}
\sphinxAtStartPar
\sphinxstylestrong{DRAN} is a data reduction and analysis software program developed to systematically reduce and analyze
\sphinxhref{http://www.hartrao.ac.za/}{HartRAO’s} drift scan data collected
by \sphinxhref{http://www.hartrao.ac.za/hh26m\_factsfile.html}{HartRAO’s 26m telescope}.
The software is a newly developed program intended to replace the old LINES program previously used at HartRAO.

\sphinxAtStartPar
The package was built on Python 3.8 and provides
a set of tools to perform common analysis tasks that
include but are not limited to,
\begin{itemize}
\item {} 
\sphinxAtStartPar
Data extraction and preperation,

\item {} 
\sphinxAtStartPar
Data modeling and fitting, as well as

\item {} 
\sphinxAtStartPar
Data visualization

\end{itemize}

\begin{sphinxadmonition}{caution}{Caution:}
\sphinxAtStartPar
This project is under active development
\end{sphinxadmonition}


\chapter{How does it work ?}
\label{\detokenize{index:how-does-it-work}}
\sphinxAtStartPar
DRAN uses a simple workflow which can be summarised as follows.
When a file is read\sphinxhyphen{}in, the program creates a dictionary where
all the extracted and calculated source parmeters are stored.

\sphinxAtStartPar
It then proceeds to
\begin{itemize}
\item {} 
\sphinxAtStartPar
cleans or filter out any {\hyperref[\detokenize{extras/rfi::doc}]{\sphinxcrossref{\DUrole{doc}{Radio frequency interference (RFI)}}}} present in the data

\item {} 
\sphinxAtStartPar
correct or remove any drift that may be present in the data due to telescope effects, and

\item {} 
\sphinxAtStartPar
tries to fit the peak of the signal ( i.e. upper 20 \sphinxhyphen{} 50\% of the beam) depending on how good the signal is after cleansing.

\end{itemize}

\sphinxAtStartPar
The result and other statistics are then written
into a database where they can then be analysed using various statistical tools.
All accompanying plots are stored in the plots folder created in the users current working directory.

\begin{sphinxadmonition}{note}{Note:}
\sphinxAtStartPar
Although data is primnarily stored in an SQLite database, it can also easily be converted into
a csv file.
\end{sphinxadmonition}

\sphinxAtStartPar
The main mode of data processing in DRAN is through the {\hyperref[\detokenize{extras/cli::doc}]{\sphinxcrossref{\DUrole{doc}{The command line interface (CLI)}}}}. This interface is best suited for
automated or semi\sphinxhyphen{}automated analysis. For a more hands on approach a
{\hyperref[\detokenize{extras/gui::doc}]{\sphinxcrossref{\DUrole{doc}{The graphical user interface (GUI)}}}} is also provided
and is best suited for handling single file inspection, fitting and timeseries analysis.


\chapter{What’s next}
\label{\detokenize{index:what-s-next}}
\sphinxAtStartPar
To get a better grasp of how the program works a set of {\hyperref[\detokenize{extras/tuts/tutorials::doc}]{\sphinxcrossref{\DUrole{doc}{Tutorials}}}} have been provided.
For instructions on how to install DRAN
see the {\hyperref[\detokenize{extras/installation::doc}]{\sphinxcrossref{\DUrole{doc}{Installation}}}} page. To jump right in,
a short {\hyperref[\detokenize{extras/quickstart::doc}]{\sphinxcrossref{\DUrole{doc}{Quickstart Guide}}}} guide is also provided.
Also please read through the {\hyperref[\detokenize{extras/caveats::doc}]{\sphinxcrossref{\DUrole{doc}{Caveats}}}} to keep updated
on the program limitations and assumptions made before beginning any analysis.

\sphinxstepscope


\section{Installation}
\label{\detokenize{extras/installation:installation}}\label{\detokenize{extras/installation::doc}}
\sphinxAtStartPar
The DRAN program was developed using the anaconda virtual environment.
Running the code on a virtual environment is highly recommended to avoid
package clashes,
and the following installation shows how to do this using
\sphinxhref{https://www.anaconda.com/products/individual}{anaconda}.
Anaconda is a data science toolkit with a wide variety of open\sphinxhyphen{}source packages and
and libraries under one hub for building powerful projects.

\begin{sphinxadmonition}{note}{Note:}
\sphinxAtStartPar
There is an issue with the installation of pyqt5 (the library required to run the GUI) on some machines while
loading the requirements file during program installation. Thus it is highly
recommended that the user utilizes an anaconda installation or a virtual environment of their choice
when installing the program.
Please also note that the software has only been tested on anaconda so using a different virtual
enviroment server might not work.
\end{sphinxadmonition}


\subsection{Installing and running dran in an anaconda virtual environment}
\label{\detokenize{extras/installation:installing-and-running-dran-in-an-anaconda-virtual-environment}}
\sphinxAtStartPar
First you will need to \sphinxhref{https://www.anaconda.com/products/individual}{download anaconda} and install it on your machine.
Once installed you will create the dran virtual environment and activate it
using the following commands

\begin{sphinxVerbatim}[commandchars=\\\{\}]
\PYGZdl{}\PYG{+w}{ }mkdir\PYG{+w}{ }dran
\PYGZdl{}\PYG{+w}{ }\PYG{n+nb}{cd}\PYG{+w}{ }dran
\PYGZdl{}\PYG{+w}{ }conda\PYG{+w}{ }create\PYG{+w}{ }\PYGZhy{}\PYGZhy{}name\PYG{+w}{ }dran\PYG{+w}{ }\PYGZhy{}\PYGZhy{}file\PYG{+w}{ }requests.txt
\PYGZdl{}\PYG{+w}{ }conda\PYG{+w}{ }activate\PYG{+w}{ }dran
\end{sphinxVerbatim}

\sphinxAtStartPar
Alternatively, you can use the pypi package manager

\begin{sphinxVerbatim}[commandchars=\\\{\}]
\PYGZdl{}\PYG{+w}{ }pip\PYG{+w}{ }install\PYG{+w}{ }dran
\end{sphinxVerbatim}

\sphinxAtStartPar
Once your virtual environment is activated. You can now head over
to {\hyperref[\detokenize{extras/tuts/tutorials::doc}]{\sphinxcrossref{\DUrole{doc}{Tutorials}}}} to start using DRAN.

\begin{sphinxadmonition}{note}{Note:}
\sphinxAtStartPar
The program has not been tested on pyenv or env yet. This is planned for future releases.
\end{sphinxadmonition}

\sphinxstepscope


\section{Caveats}
\label{\detokenize{extras/caveats:caveats}}\label{\detokenize{extras/caveats::doc}}
\sphinxAtStartPar
It is highly recommended that you familiarize yourself with
this page before starting any data reduction or analysis. This list here
is not an exhaustive list and will be updated on a case by case basis.

\sphinxAtStartPar
\sphinxstylestrong{Operating systems}

\sphinxAtStartPar
This code has only been tested in MacOS and Linux. There is no provision
at the moment for Windows systems.

\sphinxAtStartPar
\sphinxstylestrong{Help options}

\sphinxAtStartPar
Although there are multiple options you can select to run your data processing,
they can not all be run simultaneously. For detailed information on how to
use the options please refer to the \sphinxhref{tuts/tutorials.rst}{Tutorials} .

\sphinxAtStartPar
\sphinxstylestrong{Order of commands when running the program}

\begin{sphinxadmonition}{warning}{Warning:}
\sphinxAtStartPar
COMPLETE THIS!
\end{sphinxadmonition}

\sphinxAtStartPar
The program may fail or refuse to process the data if the order of the commands
is not placed appropriately. For example, this will run

\sphinxAtStartPar
but this

\sphinxAtStartPar
will produce the following error.

\sphinxAtStartPar
The above error showing up does not mean the program doesn’t work. Its basically a warning
telling you you need to rearrange the way you wrote your commands. What seems to work best is
writing the command for reading in the file first then all the other extra commands after that.
Not sure why this bug exists as yet.

\sphinxAtStartPar
\sphinxstylestrong{Processing data from tiered directory}

\begin{sphinxadmonition}{warning}{Warning:}
\sphinxAtStartPar
TRYING TO FIX THIS! as soon as possible
\end{sphinxadmonition}

\sphinxAtStartPar
If for instance you have file located in data/files/more\_files/file.fits . And there are
also some other fits files in data/files/ . The program will try to process all the files
in data/files/ that are .fits files first, then try to process your file in data/files/more\_files/ .
This tiered nature causes issues where the program fails to detect whether if has processed files
in data/files/more\_files/ or data/files/ leading to the program repeat fitting files that have
already been processed and completely skipping others that have not been processed at all.
This is a bug I am aware of but have not begun fixing yet. For best results make sure the file
or files you are trying to process follow the folder/{\color{red}\bfseries{}*}.fits file path format or just get the direct path
to the specific file you want.

\sphinxAtStartPar
\sphinxstylestrong{The dashboard}

\sphinxAtStartPar
When opening the dashboard, if the dashboard doesn’t open immediately
or you get a server not found error you may need to refresh the page
so that the dashboard can show up on your screen.

\sphinxAtStartPar
\sphinxstylestrong{Processed data storage}

\sphinxAtStartPar
Currently the program only stores processed data in an \sphinxhref{https://www.sqlite.org/index.html}{SQLite} database.
The CALDB database stores calibrated data, this is all data from HartRAO monitored calibrators. Contact the author
to get a list of these calibrators if you are unsure. The TARDB database stores all the processed target source
information. These are sources that do not fall into the known calibrator list. If you process a source that is
not in the HartRAO list of calibrators, this source will be treated as a target source and all the processed
information will be stored in the TARDB database. Support for MySQL has not been implemented yet but will be
in a future release.

\sphinxAtStartPar
\sphinxstylestrong{Database error on initial run}

\sphinxAtStartPar
When you run the code for the first time, because there will be no
database created and the following warning will pop up

\begin{sphinxadmonition}{warning}{Warning:}
\sphinxAtStartPar
Failed to read FILENAME column from table ‘TABLENAME’, the table either doesn’t exists or is corrupted.
\end{sphinxadmonition}

\sphinxAtStartPar
This warning will not stop the processing of data and the code will continue to run as per normal. In consequent runs this warning will no longer appear.

\sphinxAtStartPar
\sphinxstylestrong{Database analysis}

\sphinxAtStartPar
Analysis on sources default to Hydra A as the main calibration source. To
select own calibrator it s advised to use the gui timeseries editor.

\begin{sphinxadmonition}{note}{Note:}
\sphinxAtStartPar
Special cases with regards to observations exists. These cases are mainly concerned
with different projects that are done to test certain properties of the telescope.
The analyses of any calibrator contains only that data considered to be normal data. Instances
where certain things are beign tested, e.g. gain curve calibration, and these data are recorded
in the database, then these are ignored.
\end{sphinxadmonition}

\sphinxAtStartPar
\sphinxstylestrong{Memory allocation}

\sphinxAtStartPar
The program focus at the moment is to ensure the correct output is delivered to the user. This
causes problems with memory allocation when running large batch analysis processes. For MacOS
users you may get a “Your system has run out of application memory error” during processing. To
solve this problem, clear the python application and restart the analysis. Unfortunately this
problem can only be addressed in future releases.

\sphinxstepscope


\section{Quickstart Guide}
\label{\detokenize{extras/quickstart:quickstart-guide}}\label{\detokenize{extras/quickstart::doc}}
\sphinxAtStartPar
After the successful \sphinxhref{installation.rst}{installation} of the package libraries you can now start reducing the data. DRAN provides you with a variety of options to process drift scan data which are
discussed below.


\subsection{Starting the program without commands}
\label{\detokenize{extras/quickstart:starting-the-program-without-commands}}
\sphinxAtStartPar
As a beginner, it is advised to start running your analysis by basically just entering the program name without any options. This is useful as it starts the program and shows you step by step options you can select to reduce and analyse your data.

\sphinxAtStartPar
To start the program type

\begin{sphinxVerbatim}[commandchars=\\\{\}]
\PYG{n}{dran}
\end{sphinxVerbatim}

\sphinxAtStartPar
This command starts the program and displays a set of options a user
can select to begin processing drift scan data.

\begin{sphinxVerbatim}[commandchars=\\\{\}]
\PYG{c+c1}{\PYGZsh{}\PYGZsh{}\PYGZsh{}\PYGZsh{}\PYGZsh{}\PYGZsh{}\PYGZsh{}\PYGZsh{}\PYGZsh{}\PYGZsh{}\PYGZsh{}\PYGZsh{}\PYGZsh{}\PYGZsh{}\PYGZsh{}\PYGZsh{}\PYGZsh{}\PYGZsh{}\PYGZsh{}\PYGZsh{}\PYGZsh{}\PYGZsh{}\PYGZsh{}\PYGZsh{}\PYGZsh{}\PYGZsh{}\PYGZsh{}\PYGZsh{}\PYGZsh{}\PYGZsh{}\PYGZsh{}\PYGZsh{}\PYGZsh{}\PYGZsh{}\PYGZsh{}\PYGZsh{}\PYGZsh{}\PYGZsh{}\PYGZsh{}\PYGZsh{}\PYGZsh{}\PYGZsh{}\PYGZsh{}\PYGZsh{}\PYGZsh{}\PYGZsh{}\PYGZsh{}\PYGZsh{}\PYGZsh{}\PYGZsh{}\PYGZsh{}\PYGZsh{}\PYGZsh{}\PYGZsh{}\PYGZsh{}\PYGZsh{}\PYGZsh{}}
\PYG{c+c1}{\PYGZsh{}                                                                                            \PYGZsh{}}
\PYG{c+c1}{\PYGZsh{}             \PYGZsh{}\PYGZsh{}\PYGZsh{}\PYGZsh{}\PYGZsh{}\PYGZsh{}  \PYGZsh{}\PYGZsh{}\PYGZsh{}\PYGZsh{}\PYGZsh{}\PYGZsh{}  \PYGZsh{}\PYGZsh{}\PYGZsh{}\PYGZsh{}\PYGZsh{}\PYGZsh{} \PYGZsh{}    \PYGZsh{}                                \PYGZsh{}}
\PYG{c+c1}{\PYGZsh{}             \PYGZsh{}     \PYGZsh{} \PYGZsh{}    \PYGZsh{}  \PYGZsh{}    \PYGZsh{} \PYGZsh{} \PYGZsh{}  \PYGZsh{}                                \PYGZsh{}}
\PYG{c+c1}{\PYGZsh{}             \PYGZsh{}     \PYGZsh{} \PYGZsh{}\PYGZsh{}\PYGZsh{}\PYGZsh{}\PYGZsh{}   \PYGZsh{}\PYGZsh{}\PYGZsh{}\PYGZsh{}\PYGZsh{}\PYGZsh{} \PYGZsh{}  \PYGZsh{} \PYGZsh{}                                \PYGZsh{}}
\PYG{c+c1}{\PYGZsh{}             \PYGZsh{}     \PYGZsh{} \PYGZsh{}    \PYGZsh{}  \PYGZsh{}    \PYGZsh{} \PYGZsh{}   \PYGZsh{}\PYGZsh{}                                \PYGZsh{}}
\PYG{c+c1}{\PYGZsh{}             \PYGZsh{}\PYGZsh{}\PYGZsh{}\PYGZsh{}\PYGZsh{}\PYGZsh{}  \PYGZsh{}    \PYGZsh{}  \PYGZsh{}    \PYGZsh{} \PYGZsh{}    \PYGZsh{}                                \PYGZsh{}}
\PYG{c+c1}{\PYGZsh{}                                                                                            \PYGZsh{}}
\PYG{c+c1}{\PYGZsh{}\PYGZsh{}\PYGZsh{}\PYGZsh{}\PYGZsh{}\PYGZsh{}\PYGZsh{}\PYGZsh{}\PYGZsh{}\PYGZsh{}\PYGZsh{}\PYGZsh{}\PYGZsh{}\PYGZsh{}\PYGZsh{}\PYGZsh{}\PYGZsh{}\PYGZsh{}\PYGZsh{}\PYGZsh{}\PYGZsh{}\PYGZsh{}\PYGZsh{}\PYGZsh{}\PYGZsh{}\PYGZsh{}\PYGZsh{}\PYGZsh{}\PYGZsh{}\PYGZsh{}\PYGZsh{}\PYGZsh{}\PYGZsh{}\PYGZsh{}\PYGZsh{}\PYGZsh{}\PYGZsh{}\PYGZsh{}\PYGZsh{}\PYGZsh{}\PYGZsh{}\PYGZsh{}\PYGZsh{}\PYGZsh{}\PYGZsh{}\PYGZsh{}\PYGZsh{}\PYGZsh{}\PYGZsh{}\PYGZsh{}\PYGZsh{}\PYGZsh{}\PYGZsh{}\PYGZsh{}\PYGZsh{}\PYGZsh{}\PYGZsh{}}

\PYG{c+c1}{\PYGZsh{}     PROGRAM STARTED}


\PYG{c+c1}{\PYGZsh{}\PYGZsh{} SELECT OPTION:}
\PYG{o}{\PYGZhy{}}\PYG{o}{\PYGZhy{}}\PYG{o}{\PYGZhy{}}\PYG{o}{\PYGZhy{}}\PYG{o}{\PYGZhy{}}\PYG{o}{\PYGZhy{}}\PYG{o}{\PYGZhy{}}\PYG{o}{\PYGZhy{}}\PYG{o}{\PYGZhy{}}\PYG{o}{\PYGZhy{}}\PYG{o}{\PYGZhy{}}\PYG{o}{\PYGZhy{}}\PYG{o}{\PYGZhy{}}\PYG{o}{\PYGZhy{}}\PYG{o}{\PYGZhy{}}\PYG{o}{\PYGZhy{}}\PYG{o}{\PYGZhy{}}\PYG{o}{\PYGZhy{}}\PYG{o}{\PYGZhy{}}\PYG{o}{\PYGZhy{}}\PYG{o}{\PYGZhy{}}\PYG{o}{\PYGZhy{}}\PYG{o}{\PYGZhy{}}\PYG{o}{\PYGZhy{}}\PYG{o}{\PYGZhy{}}\PYG{o}{\PYGZhy{}}\PYG{o}{\PYGZhy{}}\PYG{o}{\PYGZhy{}}\PYG{o}{\PYGZhy{}}\PYG{o}{\PYGZhy{}}\PYG{o}{\PYGZhy{}}\PYG{o}{\PYGZhy{}}\PYG{o}{\PYGZhy{}}\PYG{o}{\PYGZhy{}}\PYG{o}{\PYGZhy{}}

\PYG{l+m+mf}{1.} \PYG{n}{Data} \PYG{n}{reduction}
\PYG{l+m+mf}{2.} \PYG{n}{Open} \PYG{n}{GUI}
\PYG{l+m+mf}{3.} \PYG{n}{Run} \PYG{n}{command} \PYG{n}{line} \PYG{n}{analysis}
\PYG{l+m+mf}{4.} \PYG{n}{Open} \PYG{n}{documentation}
\PYG{l+m+mf}{5.} \PYG{n}{Open} \PYG{n}{dashboard}
\PYG{l+m+mf}{6.} \PYG{n}{Exit} \PYG{n}{program}

\PYG{n}{Please} \PYG{n}{select} \PYG{n}{an} \PYG{n}{option}


\PYG{n}{Option} \PYG{n}{Selected}\PYG{p}{:}
\end{sphinxVerbatim}


\subsection{Starting the program with command line options}
\label{\detokenize{extras/quickstart:starting-the-program-with-command-line-options}}
\sphinxAtStartPar
To start the program with command line options.

\sphinxAtStartPar
The program also has an added functionality to let a user type in specific commands that can be run directly
from the comman line.
To see these commands, type the following

\begin{sphinxVerbatim}[commandchars=\\\{\}]
\PYG{n}{dran} \PYG{o}{\PYGZhy{}}\PYG{n}{h}
\end{sphinxVerbatim}

\sphinxAtStartPar
This will print out the set of options listed below,:

\begin{sphinxVerbatim}[commandchars=\\\{\}]
\PYG{n}{usage}\PYG{p}{:} \PYG{n}{dran}\PYG{o}{.}\PYG{n}{py} \PYG{p}{[}\PYG{o}{\PYGZhy{}}\PYG{n}{h}\PYG{p}{]} \PYG{p}{[}\PYG{o}{\PYGZhy{}}\PYG{n}{db} \PYG{n}{DB}\PYG{p}{]} \PYG{p}{[}\PYG{o}{\PYGZhy{}}\PYG{n}{f} \PYG{n}{F}\PYG{p}{]} \PYG{p}{[}\PYG{o}{\PYGZhy{}}\PYG{n}{force} \PYG{n}{FORCE}\PYG{p}{]} \PYG{p}{[}\PYG{o}{\PYGZhy{}}\PYG{n}{c} \PYG{n}{C}\PYG{p}{]} \PYG{p}{[}\PYG{o}{\PYGZhy{}}\PYG{n}{b} \PYG{n}{B}\PYG{p}{]}
               \PYG{p}{[}\PYG{o}{\PYGZhy{}}\PYG{n}{delete\PYGZus{}db} \PYG{n}{DELETE\PYGZus{}DB}\PYG{p}{]} \PYG{p}{[}\PYG{o}{\PYGZhy{}}\PYG{n}{mfp} \PYG{n}{MFP}\PYG{p}{]} \PYG{p}{[}\PYG{o}{\PYGZhy{}}\PYG{n}{keep} \PYG{n}{KEEP}\PYG{p}{]}
               \PYG{p}{[}\PYG{o}{\PYGZhy{}}\PYG{n}{deleteFrom} \PYG{n}{DELETEFROM}\PYG{p}{]} \PYG{p}{[}\PYG{o}{\PYGZhy{}}\PYG{n}{conv} \PYG{n}{CONV}\PYG{p}{]} \PYG{p}{[}\PYG{o}{\PYGZhy{}}\PYG{n}{saveLocs} \PYG{n}{SAVELOCS}\PYG{p}{]}
               \PYG{p}{[}\PYG{o}{\PYGZhy{}}\PYG{n}{fitTheoretical} \PYG{n}{FITTHEORETICAL}\PYG{p}{]}
               \PYG{p}{[}\PYG{o}{\PYGZhy{}}\PYG{n}{applyRFIremoval} \PYG{n}{APPLYRFIREMOVAL}\PYG{p}{]} \PYG{p}{[}\PYG{o}{\PYGZhy{}}\PYG{n}{fitLocs} \PYG{n}{FITLOCS}\PYG{p}{]}

\PYG{n}{Begin} \PYG{n}{processing} \PYG{n}{HartRAO} \PYG{n}{drift} \PYG{n}{scan} \PYG{n}{data}

\PYG{n}{optional} \PYG{n}{arguments}\PYG{p}{:}
\PYG{o}{\PYGZhy{}}\PYG{n}{h}\PYG{p}{,} \PYG{o}{\PYGZhy{}}\PYG{o}{\PYGZhy{}}\PYG{n}{help}            \PYG{n}{show} \PYG{n}{this} \PYG{n}{help} \PYG{n}{message} \PYG{o+ow}{and} \PYG{n}{exit}
\PYG{o}{\PYGZhy{}}\PYG{n}{db} \PYG{n}{DB}                \PYG{n}{turn} \PYG{n}{debugging} \PYG{n}{on} \PYG{o+ow}{or} \PYG{n}{off}\PYG{o}{.} \PYG{n}{e}\PYG{o}{.}\PYG{n}{g}\PYG{o}{.} \PYG{o}{\PYGZhy{}}\PYG{n}{db} \PYG{n}{on}\PYG{p}{,} \PYG{n}{by} \PYG{n}{default}
                        \PYG{n}{debug} \PYG{o+ow}{is} \PYG{n}{off}
\PYG{o}{\PYGZhy{}}\PYG{n}{f} \PYG{n}{F}                  \PYG{n}{process} \PYG{n}{file} \PYG{o+ow}{or} \PYG{n}{folder} \PYG{n}{at} \PYG{n}{given} \PYG{n}{path} \PYG{n}{e}\PYG{o}{.}\PYG{n}{g}\PYG{o}{.} \PYG{o}{\PYGZhy{}}\PYG{n}{f} \PYG{n}{data}\PYG{o}{/}\PYG{n}{Hydr}
                        \PYG{n}{aA\PYGZus{}13NB}\PYG{o}{/}\PYG{l+m+mi}{2019}\PYG{n}{d133\PYGZus{}16h12m15s\PYGZus{}Cont\PYGZus{}mike\PYGZus{}HYDRA\PYGZus{}A}\PYG{o}{.}\PYG{n}{fits} \PYG{o+ow}{or}
                        \PYG{o}{\PYGZhy{}}\PYG{n}{f} \PYG{n}{data}\PYG{o}{/}\PYG{n}{HydraA\PYGZus{}13NB}
\PYG{o}{\PYGZhy{}}\PYG{n}{force} \PYG{n}{FORCE}          \PYG{n}{force} \PYG{n}{fit} \PYG{n+nb}{all} \PYG{n}{drift} \PYG{n}{scans} \PYG{n}{y}\PYG{o}{/}\PYG{n}{n} \PYG{n}{e}\PYG{o}{.}\PYG{n}{g}\PYG{o}{.} \PYG{o}{\PYGZhy{}}\PYG{n}{force} \PYG{n}{y}\PYG{o}{.} \PYG{n}{Default}
                        \PYG{o+ow}{is} \PYG{n+nb}{set} \PYG{n}{to} \PYG{n}{n}
\PYG{o}{\PYGZhy{}}\PYG{n}{c} \PYG{n}{C}                  \PYG{n}{initiate} \PYG{n}{the} \PYG{n}{command} \PYG{n}{to} \PYG{n}{run} \PYG{n}{program}\PYG{o}{.} \PYG{n}{e}\PYG{o}{.}\PYG{n}{g}\PYG{o}{.} \PYG{o}{\PYGZhy{}}\PYG{n}{c} \PYG{n}{gui} \PYG{o+ow}{or} \PYG{o}{\PYGZhy{}}\PYG{n}{c}
                        \PYG{n}{analysis} \PYG{o+ow}{or} \PYG{o}{\PYGZhy{}}\PYG{n}{c} \PYG{n}{cli}
\PYG{o}{\PYGZhy{}}\PYG{n}{b} \PYG{n}{B}                  \PYG{n}{initiate} \PYG{n}{browser}\PYG{o}{.} \PYG{n}{e}\PYG{o}{.}\PYG{n}{g}\PYG{o}{.} \PYG{o}{\PYGZhy{}}\PYG{n}{b} \PYG{n}{docs} \PYG{o+ow}{or} \PYG{o}{\PYGZhy{}}\PYG{n}{b} \PYG{n}{dash}
\PYG{o}{\PYGZhy{}}\PYG{n}{delete\PYGZus{}db} \PYG{n}{DELETE\PYGZus{}DB}  \PYG{n}{delete} \PYG{n}{database} \PYG{n}{on} \PYG{n}{program} \PYG{n}{run}\PYG{o}{.} \PYG{n}{e}\PYG{o}{.}\PYG{n}{g}\PYG{o}{.} \PYG{o}{\PYGZhy{}}\PYG{n}{delete\PYGZus{}db} \PYG{n}{y} \PYG{o+ow}{or}
                        \PYG{o}{\PYGZhy{}}\PYG{n}{delete\PYGZus{}db} \PYG{n}{CALDB}\PYG{o}{.}\PYG{n}{db}
\PYG{o}{\PYGZhy{}}\PYG{n}{mfp} \PYG{n}{MFP}              \PYG{n}{multi}\PYG{o}{\PYGZhy{}}\PYG{n}{file} \PYG{n}{processing} \PYG{n}{of} \PYG{n}{data} \PYG{n}{between} \PYG{n}{two} \PYG{n}{dates}\PYG{o}{.} \PYG{n}{e}\PYG{o}{.}\PYG{n}{g}\PYG{o}{.}
                        \PYG{o}{\PYGZhy{}}\PYG{n}{mfp} \PYG{n}{fileList}\PYG{o}{.}\PYG{n}{txt}
\PYG{o}{\PYGZhy{}}\PYG{n}{keep} \PYG{n}{KEEP}            \PYG{n}{keep} \PYG{n}{original} \PYG{n}{plots} \PYG{k}{while} \PYG{n}{processing} \PYG{n}{data}\PYG{o}{.} \PYG{n}{e}\PYG{o}{.}\PYG{n}{g}\PYG{o}{.} \PYG{o}{\PYGZhy{}}\PYG{n}{keep}
                        \PYG{n}{y}
\PYG{o}{\PYGZhy{}}\PYG{n}{deleteFrom} \PYG{n}{DELETEFROM}
                        \PYG{o+ow}{in} \PYG{n}{coordination} \PYG{n}{eith} \PYG{n}{the} \PYG{n}{filename}\PYG{p}{,} \PYG{n}{use} \PYG{n}{deleteFrom} \PYG{n}{to}
                        \PYG{n}{delete} \PYG{n}{a} \PYG{n}{row} \PYG{k+kn}{from} \PYG{n+nn}{a} \PYG{n}{database} \PYG{n}{e}\PYG{o}{.}\PYG{n}{g}\PYG{o}{.} \PYG{n}{deleteFrom} \PYG{n}{CALDB}
\PYG{o}{\PYGZhy{}}\PYG{n}{conv} \PYG{n}{CONV}            \PYG{n}{convert} \PYG{n}{database} \PYG{n}{tables} \PYG{n}{to} \PYG{n}{csv}\PYG{o}{.} \PYG{n}{e}\PYG{o}{.}\PYG{n}{g}\PYG{o}{.} \PYG{o}{\PYGZhy{}}\PYG{n}{conv} \PYG{n}{CALDB}
\PYG{o}{\PYGZhy{}}\PYG{n}{saveLocs} \PYG{n}{SAVELOCS}    \PYG{n}{Save} \PYG{n}{fit} \PYG{n}{locations}\PYG{o}{.} \PYG{n}{e}\PYG{o}{.}\PYG{n}{g}\PYG{o}{.} \PYG{o}{\PYGZhy{}}\PYG{n}{saveLocs} \PYG{n}{y} \PYG{p}{,} \PYG{n}{default} \PYG{o+ow}{is} \PYG{n}{no}
\PYG{o}{\PYGZhy{}}\PYG{n}{fitTheoretical} \PYG{n}{FITTHEORETICAL}
                        \PYG{n}{Fit} \PYG{n}{theoretical} \PYG{n}{locations} \PYG{n}{of} \PYG{n}{the} \PYG{n}{baseline}\PYG{p}{,} \PYG{n}{i}\PYG{o}{.}\PYG{n}{e}\PYG{o}{.} \PYG{n}{the}
                        \PYG{n}{FNBW} \PYG{n}{locs}\PYG{o}{.} \PYG{n}{e}\PYG{o}{.}\PYG{n}{g}\PYG{o}{.} \PYG{o}{\PYGZhy{}}\PYG{n}{fitTheoretical} \PYG{n}{y} \PYG{p}{,} \PYG{n}{default} \PYG{o+ow}{is} \PYG{n}{no}\PYG{o}{.}
\PYG{o}{\PYGZhy{}}\PYG{n}{applyRFIremoval} \PYG{n}{APPLYRFIREMOVAL}
                        \PYG{n}{Apply} \PYG{o+ow}{or} \PYG{n}{don}\PYG{l+s+s1}{\PYGZsq{}}\PYG{l+s+s1}{t apply RFI removal e.g. \PYGZhy{}applyRFIremoval}
                        \PYG{n}{n}\PYG{p}{,} \PYG{n}{default} \PYG{o+ow}{is} \PYG{n}{yes}
\PYG{o}{\PYGZhy{}}\PYG{n}{fitLocs} \PYG{n}{FITLOCS}      \PYG{n}{Fit} \PYG{n}{data} \PYG{n}{at} \PYG{n}{sepcified} \PYG{n}{locations} \PYG{n}{e}\PYG{o}{.}\PYG{n}{g}\PYG{o}{.} \PYG{o}{\PYGZhy{}}\PYG{n}{fitLocs}
                        \PYG{n}{pathToFile}
\end{sphinxVerbatim}

\begin{sphinxadmonition}{warning}{Warning:}
\sphinxAtStartPar
If you enter a command that doesn’t exist the following error pops up

\sphinxAtStartPar
e.g. \$ python dran.py command\_that\_doesnt\_exists

\sphinxAtStartPar
usage: dran.py {[}\sphinxhyphen{}h{]} {[}\sphinxhyphen{}db DB{]} {[}\sphinxhyphen{}f F{]} {[}\sphinxhyphen{}force FORCE{]} {[}\sphinxhyphen{}c C{]} {[}\sphinxhyphen{}b B{]}
{[}\sphinxhyphen{}delete\_db DELETE\_DB{]} {[}\sphinxhyphen{}mfp MFP{]} {[}\sphinxhyphen{}keep KEEP{]}
{[}\sphinxhyphen{}deleteFrom DELETEFROM{]} {[}\sphinxhyphen{}conv CONV{]} {[}\sphinxhyphen{}saveLocs SAVELOCS{]}
{[}\sphinxhyphen{}fitTheoretical FITTHEORETICAL{]}
{[}\sphinxhyphen{}applyRFIremoval APPLYRFIREMOVAL{]} {[}\sphinxhyphen{}fitLocs FITLOCS{]}

\sphinxAtStartPar
dran.py: error: unrecognized arguments: command\_that\_doesnt\_exists
\end{sphinxadmonition}

\sphinxAtStartPar
\#\#\# For example:

\sphinxAtStartPar
To perform data reduction on a single file, type in

\sphinxAtStartPar
dran

\sphinxAtStartPar
This will bring up the menu shown below with options to select how you want to proceeed with your data reduction or analysis.

\sphinxAtStartPar
In our case we would select option 1 as we are looking to get an automated reduction.

\sphinxAtStartPar
\#\#\#\# Option 1

\sphinxAtStartPar
Upon selecting option 1, the following propmt will be displayed

\sphinxAtStartPar
Here you will be shown 3 more options.
\sphinxhyphen{} Option 1 : reduce 1 file
\sphinxhyphen{} Option 2 : reduce a folder containg files of 1 source at a particular frequency
\sphinxhyphen{} Option 3 : reduce all files in the directory where all of your sources are located.

\sphinxAtStartPar
In our case we will select option 1 again as we are only looking to process a single file. From here a gui will pop up asking you to locate the file you want to process. Once the file has been located, the program will automatically process and fit your data and make plots of the fit which will be locate in the plots folder. Congratulations, you have just ran your first data reduction.

\sphinxAtStartPar
\#\#\#\# Option 2

\sphinxAtStartPar
Allows the user to manually reduce the data. With this option the user can tinker with fitting settings and options to get the best fit possible for what they are working on.

\sphinxAtStartPar
\#\#\#\# Option 3

\sphinxAtStartPar
Opens up the the GUI.

\sphinxAtStartPar
\#\#\#\# Option 4

\sphinxAtStartPar
Opens up the web resource. The resource contains the code documentation as well as \sphinxstylestrong{step by step tutorials} and results from the test data.

\sphinxAtStartPar
\#\#\#\# Option 0

\sphinxAtStartPar
Exits the program

\sphinxAtStartPar
As shown above some options contain follow up options to further refine your processing selection. Its ultimately up to the user to decide how they want to process their data.

\sphinxAtStartPar
\#\#\# More examples.

\sphinxAtStartPar
To perform autotomated data reduction on a single file given the absolute path

\sphinxAtStartPar
dran \sphinxhyphen{}f filepath

\sphinxAtStartPar
i.e.
dran \sphinxhyphen{}f /Users/pfesesanivanzyl/Final\sphinxhyphen{}PhD/code/all\_hartrao\_data/HydraA\_13NB/2014d047\_20h30m12s\_Cont\_mike\_HYDRA\_A.fits

\sphinxAtStartPar
To perform data reduction on a single file with debugging

\sphinxAtStartPar
dran \sphinxhyphen{}f filepath \sphinxhyphen{}db y

\sphinxAtStartPar
i.e.
dran \sphinxhyphen{}f /Users/pfesesanivanzyl/Final\sphinxhyphen{}PhD/code/all\_hartrao\_data/HydraA\_13NB/2014d047\_20h30m12s\_Cont\_mike\_HYDRA\_A.fits \sphinxhyphen{}db y

\sphinxAtStartPar
To perform data reduction/analysis using the GUI
dran \sphinxhyphen{}g gui

\sphinxAtStartPar
Once the GUI landing page pops up select the “Edit Driftscan” option.

\sphinxAtStartPar
For a full tutorial please view the web resource on option 4.

\sphinxstepscope


\section{The command line interface (CLI)}
\label{\detokenize{extras/cli:the-command-line-interface-cli}}\label{\detokenize{extras/cli::doc}}
\sphinxAtStartPar
In this mode data can be processed in either automated or manual
mode. In automated mode, the user has the option to process the data
using the predefined data reduction method “dran\sphinxhyphen{}auto” using the prompt

\begin{sphinxVerbatim}[commandchars=\\\{\}]
\PYGZdl{} dran\PYGZhy{}auto path\PYGZus{}to\PYGZus{}file

or

\PYGZdl{} dran\PYGZhy{}auto path\PYGZus{}to\PYGZus{}folder
\end{sphinxVerbatim}

\sphinxAtStartPar
This option automatically selects and locates all the required fitting
parameters including the positions and locations of the baseline blocks
and the position of the peak. The default fitting methods for the data
performs a 1st order fit to the baseline blocks and a 2nd order polynomial
fit for the peak. This processing method can also be semi\sphinxhyphen{}automated allowing for
minor changes which can be found in the help resource which can be accessed using the
following prompt

\begin{sphinxVerbatim}[commandchars=\\\{\}]
\PYG{o}{\PYGZpc{}} \PYG{n}{dran}\PYG{o}{\PYGZhy{}}\PYG{n}{auto} \PYG{o}{\PYGZhy{}}\PYG{n}{h}

\PYG{n}{usage}\PYG{p}{:} \PYG{n}{DRAN}\PYG{o}{\PYGZhy{}}\PYG{n}{AUTO} \PYG{p}{[}\PYG{o}{\PYGZhy{}}\PYG{n}{h}\PYG{p}{]} \PYG{p}{[}\PYG{o}{\PYGZhy{}}\PYG{n}{db} \PYG{n}{DB}\PYG{p}{]} \PYG{p}{[}\PYG{o}{\PYGZhy{}}\PYG{n}{f} \PYG{n}{F}\PYG{p}{]} \PYG{p}{[}\PYG{o}{\PYGZhy{}}\PYG{n}{delete\PYGZus{}db} \PYG{n}{DELETE\PYGZus{}DB}\PYG{p}{]} \PYG{p}{[}\PYG{o}{\PYGZhy{}}\PYG{n}{conv} \PYG{n}{CONV}\PYG{p}{]}
                \PYG{p}{[}\PYG{o}{\PYGZhy{}}\PYG{n}{quickview} \PYG{n}{QUICKVIEW}\PYG{p}{]} \PYG{p}{[}\PYG{o}{\PYGZhy{}}\PYG{o}{\PYGZhy{}}\PYG{n}{version}\PYG{p}{]}

\PYG{n}{Begin} \PYG{n}{processing} \PYG{n}{HartRAO} \PYG{n}{drift} \PYG{n}{scan} \PYG{n}{data}

\PYG{n}{optional} \PYG{n}{arguments}\PYG{p}{:}
  \PYG{o}{\PYGZhy{}}\PYG{n}{h}\PYG{p}{,} \PYG{o}{\PYGZhy{}}\PYG{o}{\PYGZhy{}}\PYG{n}{help}              \PYG{n}{show} \PYG{n}{this} \PYG{n}{help} \PYG{n}{message} \PYG{o+ow}{and} \PYG{n}{exit}
  \PYG{o}{\PYGZhy{}}\PYG{n}{db} \PYG{n}{DB}                  \PYG{n}{turn} \PYG{n}{debugging} \PYG{n}{on} \PYG{o+ow}{or} \PYG{n}{off}\PYG{o}{.} \PYG{n}{e}\PYG{o}{.}\PYG{n}{g}\PYG{o}{.} \PYG{o}{\PYGZhy{}}\PYG{n}{db} \PYG{n}{on}\PYG{p}{,} \PYG{n}{by} \PYG{n}{default} \PYG{n}{debug} \PYG{o+ow}{is} \PYG{n}{off}
  \PYG{o}{\PYGZhy{}}\PYG{n}{f} \PYG{n}{F}                    \PYG{n}{process} \PYG{n}{file} \PYG{o+ow}{or} \PYG{n}{folder} \PYG{n}{at} \PYG{n}{given} \PYG{n}{path} \PYG{n}{e}\PYG{o}{.}\PYG{n}{g}\PYG{o}{.} \PYG{o}{\PYGZhy{}}\PYG{n}{f}
                          \PYG{n}{data}\PYG{o}{/}\PYG{n}{HydraA\PYGZus{}13NB}\PYG{o}{/}\PYG{l+m+mi}{2019}\PYG{n}{d133\PYGZus{}16h12m15s\PYGZus{}Cont\PYGZus{}mike\PYGZus{}HYDRA\PYGZus{}A}\PYG{o}{.}\PYG{n}{fits} \PYG{o+ow}{or} \PYG{o}{\PYGZhy{}}\PYG{n}{f}
                          \PYG{n}{data}\PYG{o}{/}\PYG{n}{HydraA\PYGZus{}13NB}
  \PYG{o}{\PYGZhy{}}\PYG{n}{delete\PYGZus{}db} \PYG{n}{DELETE\PYGZus{}DB}    \PYG{n}{delete} \PYG{n}{database} \PYG{n}{on} \PYG{n}{program} \PYG{n}{run}\PYG{o}{.} \PYG{n}{e}\PYG{o}{.}\PYG{n}{g}\PYG{o}{.} \PYG{o}{\PYGZhy{}}\PYG{n}{delete\PYGZus{}db} \PYG{n+nb}{all} \PYG{o+ow}{or} \PYG{o}{\PYGZhy{}}\PYG{n}{delete\PYGZus{}db} \PYG{n}{CALDB}\PYG{o}{.}\PYG{n}{db}
  \PYG{o}{\PYGZhy{}}\PYG{n}{conv} \PYG{n}{CONV}              \PYG{n}{convert} \PYG{n}{database} \PYG{n}{tables} \PYG{n}{to} \PYG{n}{csv}\PYG{o}{.} \PYG{n}{e}\PYG{o}{.}\PYG{n}{g}\PYG{o}{.} \PYG{o}{\PYGZhy{}}\PYG{n}{conv} \PYG{n}{CALDB}
  \PYG{o}{\PYGZhy{}}\PYG{n}{quickview} \PYG{n}{QUICKVIEW}    \PYG{n}{get} \PYG{n}{quickview} \PYG{n}{of} \PYG{n}{data} \PYG{n}{e}\PYG{o}{.}\PYG{n}{g}\PYG{o}{.} \PYG{o}{\PYGZhy{}}\PYG{n}{quickview} \PYG{n}{y}
  \PYG{o}{\PYGZhy{}}\PYG{o}{\PYGZhy{}}\PYG{n}{version}               \PYG{n}{show} \PYG{n}{program}\PYG{l+s+s1}{\PYGZsq{}}\PYG{l+s+s1}{s version number and exit}
\end{sphinxVerbatim}

\sphinxAtStartPar
On the other hand manual mode gives the user total autonomy on the drift scan
reduction process. However, this implementation has not been activated yet.

\sphinxAtStartPar
The following tutorial will show you how to process data in semi\sphinxhyphen{}automation.


\subsection{automated data reduction}
\label{\detokenize{extras/cli:automated-data-reduction}}
\sphinxAtStartPar
Semi\sphinxhyphen{}automated data reduction involves typing in commands to
run certain types of analysis using the program. This is the
prefered mode of data reduction as it caters for both single
and batch mode data analysis.

\sphinxAtStartPar
Before starting any analysis it is recommended that you first
read the help doc to familiarize yourself with the basic commands
required to perform data analysis with DRAN. This is done using

\begin{sphinxVerbatim}[commandchars=\\\{\}]
\PYGZdl{}\PYG{+w}{ }python\PYG{+w}{ }dran.py\PYG{+w}{ }\PYGZhy{}h
\end{sphinxVerbatim}

\sphinxAtStartPar
The above line of code outputs the following:

\begin{sphinxVerbatim}[commandchars=\\\{\},numbers=left,firstnumber=1,stepnumber=1]
usage:\PYG{+w}{ }dran.py\PYG{+w}{ }\PYG{o}{[}\PYGZhy{}h\PYG{o}{]}\PYG{+w}{ }\PYG{o}{[}\PYGZhy{}db\PYG{+w}{ }DB\PYG{o}{]}\PYG{+w}{ }\PYG{o}{[}\PYGZhy{}f\PYG{+w}{ }F\PYG{o}{]}\PYG{+w}{ }\PYG{o}{[}\PYGZhy{}force\PYG{+w}{ }FORCE\PYG{o}{]}\PYG{+w}{ }\PYG{o}{[}\PYGZhy{}c\PYG{+w}{ }C\PYG{o}{]}\PYG{+w}{ }\PYG{o}{[}\PYGZhy{}b\PYG{+w}{ }B\PYG{o}{]}\PYG{+w}{ }\PYG{o}{[}\PYGZhy{}delete\PYGZus{}db\PYG{+w}{ }DELETE\PYGZus{}DB\PYG{o}{]}\PYG{+w}{ }\PYG{o}{[}\PYGZhy{}mfp\PYG{+w}{ }MFP\PYG{o}{]}
\PYG{o}{[}\PYGZhy{}keep\PYG{+w}{ }KEEP\PYG{o}{]}\PYG{+w}{ }\PYG{o}{[}\PYGZhy{}delete\PYGZus{}from\PYG{+w}{ }DELETE\PYGZus{}FROM\PYG{o}{]}\PYG{+w}{ }\PYG{o}{[}\PYGZhy{}conv\PYG{+w}{  }CONV\PYG{o}{]}

Begin\PYG{+w}{ }processing\PYG{+w}{ }HartRAO\PYG{+w}{ }drift\PYG{+w}{ }scan\PYG{+w}{ }data

optional\PYG{+w}{ }arguments:
\PYG{+w}{  }\PYGZhy{}h,\PYG{+w}{ }\PYGZhy{}\PYGZhy{}help\PYG{+w}{            }show\PYG{+w}{ }this\PYG{+w}{ }\PYG{n+nb}{help}\PYG{+w}{ }message\PYG{+w}{ }and\PYG{+w}{ }\PYG{n+nb}{exit}
\PYG{+w}{  }\PYGZhy{}db\PYG{+w}{ }DB\PYG{+w}{                }turn\PYG{+w}{ }debugging\PYG{+w}{ }on\PYG{+w}{ }or\PYG{+w}{ }off.\PYG{+w}{ }e.g.\PYG{+w}{ }\PYGZhy{}db\PYG{+w}{ }on,\PYG{+w}{ }by\PYG{+w}{ }default
\PYG{+w}{                      }debug\PYG{+w}{ }is\PYG{+w}{ }off
\PYG{+w}{  }\PYGZhy{}f\PYG{+w}{ }F\PYG{+w}{                  }process\PYG{+w}{ }file\PYG{+w}{ }or\PYG{+w}{ }folder\PYG{+w}{ }at\PYG{+w}{ }given\PYG{+w}{ }path\PYG{+w}{ }e.g.\PYG{+w}{ }\PYGZhy{}f\PYG{+w}{ }data/Hydr
\PYG{+w}{                      }aA\PYGZus{}13NB/2019d133\PYGZus{}16h12m15s\PYGZus{}Cont\PYGZus{}mike\PYGZus{}HYDRA\PYGZus{}A.fits\PYG{+w}{ }or
\PYG{+w}{                      }\PYGZhy{}f\PYG{+w}{ }data/HydraA\PYGZus{}13NB\PYG{+w}{ }or\PYG{+w}{ }\PYGZhy{}f\PYG{+w}{ }data/
\PYG{+w}{  }\PYGZhy{}force\PYG{+w}{ }FORCE\PYG{+w}{          }force\PYG{+w}{ }fit\PYG{+w}{ }all\PYG{+w}{ }drift\PYG{+w}{ }scans\PYG{+w}{ }y/n\PYG{+w}{ }e.g.\PYG{+w}{ }\PYGZhy{}force\PYG{+w}{ }y.\PYG{+w}{ }Default
\PYG{+w}{                      }is\PYG{+w}{ }\PYG{n+nb}{set}\PYG{+w}{ }to\PYG{+w}{ }n
\PYG{+w}{  }\PYGZhy{}c\PYG{+w}{ }C\PYG{+w}{                  }initiate\PYG{+w}{ }the\PYG{+w}{ }\PYG{n+nb}{command}\PYG{+w}{ }to\PYG{+w}{ }run\PYG{+w}{ }program.\PYG{+w}{ }e.g.\PYG{+w}{ }\PYGZhy{}c\PYG{+w}{ }gui\PYG{+w}{ }or\PYG{+w}{ }\PYGZhy{}c
\PYG{+w}{                      }run\PYGZus{}auto\PYGZus{}analysis\PYG{+w}{ }or\PYG{+w}{ }\PYGZhy{}c\PYG{+w}{ }cmdl
\PYG{+w}{  }\PYGZhy{}b\PYG{+w}{ }B\PYG{+w}{                  }initiate\PYG{+w}{ }browser.\PYG{+w}{ }e.g.\PYG{+w}{ }\PYGZhy{}b\PYG{+w}{ }docs\PYG{+w}{ }or\PYG{+w}{ }\PYGZhy{}b\PYG{+w}{ }dash
\PYG{+w}{  }\PYGZhy{}delete\PYGZus{}db\PYG{+w}{ }DELETE\PYGZus{}DB\PYG{+w}{  }delete\PYG{+w}{ }database\PYG{+w}{ }on\PYG{+w}{ }program\PYG{+w}{ }run.\PYG{+w}{ }e.g.\PYG{+w}{ }\PYGZhy{}delete\PYGZus{}db\PYG{+w}{ }y
\PYG{+w}{  }\PYGZhy{}mfp\PYG{+w}{ }MFP\PYG{+w}{              }multi\PYGZhy{}file\PYG{+w}{ }processing\PYG{+w}{ }of\PYG{+w}{ }data\PYG{+w}{ }between\PYG{+w}{ }two\PYG{+w}{ }dates.\PYG{+w}{ }e.g.
\PYG{+w}{                      }\PYGZhy{}mfp\PYG{+w}{ }fileList.txt
\PYG{+w}{  }\PYGZhy{}keep\PYG{+w}{ }KEEP\PYG{+w}{            }keep\PYG{+w}{ }original\PYG{+w}{ }plots\PYG{+w}{ }\PYG{k}{while}\PYG{+w}{ }processing\PYG{+w}{ }data.\PYG{+w}{ }e.g.\PYG{+w}{ }\PYGZhy{}keep\PYG{+w}{ }y
\PYG{+w}{  }\PYGZhy{}delete\PYGZus{}from\PYG{+w}{ }DELETE\PYGZus{}FROM
\PYG{+w}{                        }\PYG{k}{in}\PYG{+w}{ }coordination\PYG{+w}{ }eith\PYG{+w}{ }the\PYG{+w}{ }filename,\PYG{+w}{ }use\PYG{+w}{ }delete\PYGZus{}from\PYG{+w}{ }to\PYG{+w}{ }delete\PYG{+w}{ }a\PYG{+w}{ }row\PYG{+w}{ }from\PYG{+w}{ }a\PYG{+w}{ }database
\PYG{+w}{                        }e.g.\PYG{+w}{ }delete\PYGZus{}from\PYG{+w}{ }CALDB
\PYG{+w}{  }\PYGZhy{}conv\PYG{+w}{  }CONV\PYG{+w}{           }convert\PYG{+w}{ }the\PYG{+w}{ }database\PYG{+w}{ }tables\PYG{+w}{ }to\PYG{+w}{ }csv.\PYG{+w}{ }e.g.\PYG{+w}{ }conv\PYG{+w}{ }y
\end{sphinxVerbatim}

\sphinxAtStartPar
Depending on the process you want to run, you can select one or
more of the available options. If there are more options you would
like implemented please email the author.

\sphinxAtStartPar
To perform an automated data reduction process on a single file

\begin{sphinxVerbatim}[commandchars=\\\{\}]
\PYGZdl{}\PYG{+w}{ }python\PYG{+w}{ }dran.py\PYG{+w}{ }\PYGZhy{}f\PYG{+w}{ }path\PYGZhy{}to\PYGZhy{}file/filename.fits
\end{sphinxVerbatim}

\sphinxAtStartPar
if you want to set debuggin on

\begin{sphinxVerbatim}[commandchars=\\\{\}]
\PYGZdl{}\PYG{+w}{ }python\PYG{+w}{ }dran.py\PYG{+w}{ }\PYGZhy{}f\PYG{+w}{ }path\PYGZhy{}to\PYGZhy{}file/filename.fits\PYG{+w}{ }\PYGZhy{}db\PYG{+w}{ }on
\end{sphinxVerbatim}

\sphinxAtStartPar
To force a fit on all drift scans, especially those that the
program would generally categorize as bad scans and not fit them

\begin{sphinxVerbatim}[commandchars=\\\{\}]
\PYGZdl{}\PYG{+w}{ }python\PYG{+w}{ }dran.py\PYG{+w}{ }\PYGZhy{}f\PYG{+w}{ }path\PYGZhy{}to\PYGZhy{}file/filename.fits\PYG{+w}{ }\PYGZhy{}db\PYG{+w}{ }on\PYG{+w}{ }\PYGZhy{}force\PYG{+w}{ }y
\end{sphinxVerbatim}

\sphinxAtStartPar
To delete both the calibration and target databases everytime the
program starts

\begin{sphinxVerbatim}[commandchars=\\\{\}]
\PYGZdl{}\PYG{+w}{ }python\PYG{+w}{ }dran.py\PYG{+w}{ }\PYGZhy{}f\PYG{+w}{ }path\PYGZhy{}to\PYGZhy{}file/filename.fits\PYG{+w}{ }\PYGZhy{}delete\PYGZus{}db\PYG{+w}{ }y
\end{sphinxVerbatim}

\sphinxAtStartPar
To process multiple chunks of data within seperate periods of the
year you use the following command

\begin{sphinxVerbatim}[commandchars=\\\{\}]
\PYGZdl{}\PYG{+w}{ }python\PYG{+w}{ }dran.py\PYG{+w}{ }\PYGZhy{}mfp\PYG{+w}{ }file\PYGZus{}list.txt
\end{sphinxVerbatim}

\begin{sphinxadmonition}{note}{Note:}
\sphinxAtStartPar
This assumes that there exists a file called
fileList.txt in the current directory that has the full path
to the folder containing the source fits files you want to
process. This file also has a start and end date in “YYYYdDDD”
format stipulating the data range you want to process.
\end{sphinxadmonition}

\sphinxAtStartPar
To process all the data located in your directory, this is the
directory that contains all the folders containing your fits files

\begin{sphinxVerbatim}[commandchars=\\\{\}]
\PYGZdl{}\PYG{+w}{ }python\PYG{+w}{ }dran.py\PYG{+w}{ }\PYGZhy{}f\PYG{+w}{ }path\PYGZhy{}to\PYGZhy{}directory
\end{sphinxVerbatim}

\sphinxAtStartPar
To run data reduction using the GUI

\begin{sphinxVerbatim}[commandchars=\\\{\}]
\PYGZdl{}\PYG{+w}{ }python\PYG{+w}{ }dran.py\PYG{+w}{ }\PYGZhy{}c\PYG{+w}{ }gui
\end{sphinxVerbatim}

\sphinxAtStartPar
To run data reduction using the GUI with a pre\sphinxhyphen{}selected file

\begin{sphinxVerbatim}[commandchars=\\\{\}]
\PYGZdl{}\PYG{+w}{ }python\PYG{+w}{ }dran.py\PYG{+w}{ }\PYGZhy{}c\PYG{+w}{ }gui\PYG{+w}{ }\PYGZhy{}f\PYG{+w}{ }path\PYGZhy{}to\PYGZhy{}file
\end{sphinxVerbatim}

\sphinxAtStartPar
To initiate the browser to view the analyzed data on a dashboard

\begin{sphinxVerbatim}[commandchars=\\\{\}]
\PYGZdl{}\PYG{+w}{ }python\PYG{+w}{ }dran.py\PYG{+w}{ }\PYGZhy{}f\PYG{+w}{ }path\PYGZhy{}to\PYGZhy{}file/filename.fits\PYG{+w}{ }\PYGZhy{}b\PYG{+w}{ }dash
\end{sphinxVerbatim}

\sphinxAtStartPar
To view the web documentation guide of the software

\begin{sphinxVerbatim}[commandchars=\\\{\}]
\PYGZdl{}\PYG{+w}{ }python\PYG{+w}{ }dran.py\PYG{+w}{ }\PYGZhy{}f\PYG{+w}{ }path\PYGZhy{}to\PYGZhy{}file/filename.fits\PYG{+w}{ }\PYGZhy{}b\PYG{+w}{ }docs
\end{sphinxVerbatim}

\sphinxAtStartPar
To run the semi\sphinxhyphen{}automated data reduction process through the
command line

\begin{sphinxVerbatim}[commandchars=\\\{\}]
\PYGZdl{}\PYG{+w}{ }python\PYG{+w}{ }dran.py
\end{sphinxVerbatim}

\sphinxAtStartPar
and follow the prompts. This will bring up the menu shown below
with options to select how you want to proceeed with your data
reduction or analysis.

\begin{sphinxVerbatim}[commandchars=\\\{\},numbers=left,firstnumber=1,stepnumber=1]
\PYGZsh{}\PYGZsh{}\PYGZsh{}\PYGZsh{}\PYGZsh{}\PYGZsh{}\PYGZsh{}\PYGZsh{}\PYGZsh{}\PYGZsh{}\PYGZsh{}\PYGZsh{}\PYGZsh{}\PYGZsh{}\PYGZsh{}\PYGZsh{}\PYGZsh{}\PYGZsh{}\PYGZsh{}\PYGZsh{}\PYGZsh{}\PYGZsh{}\PYGZsh{}\PYGZsh{}\PYGZsh{}\PYGZsh{}\PYGZsh{}\PYGZsh{}\PYGZsh{}\PYGZsh{}\PYGZsh{}\PYGZsh{}\PYGZsh{}\PYGZsh{}\PYGZsh{}\PYGZsh{}\PYGZsh{}\PYGZsh{}\PYGZsh{}\PYGZsh{}\PYGZsh{}\PYGZsh{}\PYGZsh{}\PYGZsh{}\PYGZsh{}\PYGZsh{}\PYGZsh{}\PYGZsh{}\PYGZsh{}\PYGZsh{}\PYGZsh{}\PYGZsh{}\PYGZsh{}\PYGZsh{}\PYGZsh{}\PYGZsh{}\PYGZsh{}\PYGZsh{}\PYGZsh{}\PYGZsh{}\PYGZsh{}\PYGZsh{}\PYGZsh{}\PYGZsh{}\PYGZsh{}\PYGZsh{}
\PYGZsh{}                                                                                              \PYGZsh{}
\PYGZsh{}              \PYGZsh{}\PYGZsh{}\PYGZsh{}\PYGZsh{}\PYGZsh{}\PYGZsh{}  \PYGZsh{}\PYGZsh{}\PYGZsh{}\PYGZsh{}\PYGZsh{}\PYGZsh{}  \PYGZsh{}\PYGZsh{}\PYGZsh{}\PYGZsh{}\PYGZsh{}\PYGZsh{} \PYGZsh{}    \PYGZsh{}                                   \PYGZsh{}
\PYGZsh{}              \PYGZsh{}     \PYGZsh{} \PYGZsh{}    \PYGZsh{}  \PYGZsh{}    \PYGZsh{} \PYGZsh{} \PYGZsh{}  \PYGZsh{}                                   \PYGZsh{}
\PYGZsh{}              \PYGZsh{}     \PYGZsh{} \PYGZsh{}\PYGZsh{}\PYGZsh{}\PYGZsh{}\PYGZsh{}   \PYGZsh{}\PYGZsh{}\PYGZsh{}\PYGZsh{}\PYGZsh{}\PYGZsh{} \PYGZsh{}  \PYGZsh{} \PYGZsh{}                                   \PYGZsh{}
\PYGZsh{}              \PYGZsh{}     \PYGZsh{} \PYGZsh{}    \PYGZsh{}  \PYGZsh{}    \PYGZsh{} \PYGZsh{}   \PYGZsh{}\PYGZsh{}                                   \PYGZsh{}
\PYGZsh{}              \PYGZsh{}\PYGZsh{}\PYGZsh{}\PYGZsh{}\PYGZsh{}\PYGZsh{}  \PYGZsh{}    \PYGZsh{}  \PYGZsh{}    \PYGZsh{} \PYGZsh{}    \PYGZsh{}                                   \PYGZsh{}
\PYGZsh{}                                                                                              \PYGZsh{}
\PYGZsh{}\PYGZsh{}\PYGZsh{}\PYGZsh{}\PYGZsh{}\PYGZsh{}\PYGZsh{}\PYGZsh{}\PYGZsh{}\PYGZsh{}\PYGZsh{}\PYGZsh{}\PYGZsh{}\PYGZsh{}\PYGZsh{}\PYGZsh{}\PYGZsh{}\PYGZsh{}\PYGZsh{}\PYGZsh{}\PYGZsh{}\PYGZsh{}\PYGZsh{}\PYGZsh{}\PYGZsh{}\PYGZsh{}\PYGZsh{}\PYGZsh{}\PYGZsh{}\PYGZsh{}\PYGZsh{}\PYGZsh{}\PYGZsh{}\PYGZsh{}\PYGZsh{}\PYGZsh{}\PYGZsh{}\PYGZsh{}\PYGZsh{}\PYGZsh{}\PYGZsh{}\PYGZsh{}\PYGZsh{}\PYGZsh{}\PYGZsh{}\PYGZsh{}\PYGZsh{}\PYGZsh{}\PYGZsh{}\PYGZsh{}\PYGZsh{}\PYGZsh{}\PYGZsh{}\PYGZsh{}\PYGZsh{}\PYGZsh{}\PYGZsh{}\PYGZsh{}\PYGZsh{}\PYGZsh{}\PYGZsh{}\PYGZsh{}\PYGZsh{}\PYGZsh{}\PYGZsh{}\PYGZsh{}

\PYGZsh{}      PROGRAM STARTED


\PYGZsh{}\PYGZsh{} SELECT OPTION:
\PYGZhy{}\PYGZhy{}\PYGZhy{}\PYGZhy{}\PYGZhy{}\PYGZhy{}\PYGZhy{}\PYGZhy{}\PYGZhy{}\PYGZhy{}\PYGZhy{}\PYGZhy{}\PYGZhy{}\PYGZhy{}\PYGZhy{}\PYGZhy{}\PYGZhy{}\PYGZhy{}\PYGZhy{}\PYGZhy{}\PYGZhy{}\PYGZhy{}\PYGZhy{}\PYGZhy{}\PYGZhy{}\PYGZhy{}\PYGZhy{}\PYGZhy{}\PYGZhy{}\PYGZhy{}\PYGZhy{}\PYGZhy{}\PYGZhy{}\PYGZhy{}\PYGZhy{}

 1. Data reduction
 2. Open GUI
 3. Run command line analysis
 4. Open documentation
 5. Open dashboard
 6. Exit program

 Please select an option

 Option Selected:
\end{sphinxVerbatim}

\sphinxAtStartPar
To process data using the gui

\begin{sphinxVerbatim}[commandchars=\\\{\}]
\PYGZdl{}\PYG{+w}{ }python\PYG{+w}{ }dran.py\PYG{+w}{ }\PYGZhy{}f\PYG{+w}{ }path\PYGZhy{}to\PYGZhy{}file/filename.fits\PYG{+w}{ }\PYGZhy{}c\PYG{+w}{ }gui
\end{sphinxVerbatim}


\subsection{Manual data reduction}
\label{\detokenize{extras/cli:manual-data-reduction}}
\sphinxAtStartPar
This has been implemented but not complete for testing yet.

\sphinxstepscope


\section{The graphical user interface (GUI)}
\label{\detokenize{extras/gui:the-graphical-user-interface-gui}}\label{\detokenize{extras/gui::doc}}
\sphinxAtStartPar
The graphical user interface lets the user manually examine a drift scan.
This is done through a graphical interface that enables a user to select
specific locations on the drift scan they would like to fit. The gui also
allows the user to view changes made to a drift scan and adjust them
accordingly.  A walk through of the GUI is given below.


\section{The landing page}
\label{\detokenize{extras/gui:the-landing-page}}
\sphinxAtStartPar
The first page you see when the GUI is initiated is the landing page.
This page contains the different options the GUI offers with regards to viewing
and processing your data. You can edit your drift scan (e.g. perform fitting),
edit your timeseries (e.g. manually remove outliers) and view plots of the
processed data. A brief summary of each of these options is given below.

\noindent\sphinxincludegraphics{{landing}.png}


\section{Edit driftscan}
\label{\detokenize{extras/gui:edit-driftscan}}
\sphinxAtStartPar
The “Edit Driftscan” button opens up the drift scan editting window.
This page shows the basic layout of the GUI. On the left is the “Scan properties”
tab, this tab consists of a curated list of some of the drift scan properties found in
the drift scan fits file on the observed source. On the second tab, “Fit window”, is
information used for the actual fitting. This is currently hidden here but will be
shown in the next image. We also have a section to display all the text
outputs resulting from running operations on the program, as well as buttons to
open a file, view, reset and save drift scan fitting information. On the right is the
plot window where the drift scan data is displayed. The next plot shows what the GUI
looks like once a drift scan file is opened.

\noindent\sphinxincludegraphics{{home}.png}

\sphinxAtStartPar
Upon opening a file, the drift scan data is loaded onto the plot window. The top plot
displays the current drift scan, and the bottom one is a place holder for the residuals.
This operation also toggles the display to the “Fit window” mentioned previously. On this
window is the information we need to fit the beam/beams of the drift scan. To toggle between
different drift scans we use the “curren plot selection” toggle.

\noindent\sphinxincludegraphics{{loaded_scan}.png}

\sphinxAtStartPar
To beginning fitting the drift scan, one can click on the drift scan at the location at
which the fit is to be done as shown in the following image.

\noindent\sphinxincludegraphics{{base_select}.png}

\sphinxAtStartPar
The code is hot wired to process fitting the baseline and peak differently. This is
controlled by the “Fit Location” toggle. If you want to fit the baseline, the toggle
needs to be on “Base”. Then once you are happy with the locations you clicked on or
selected on the plot, you use the “Fit data” button to initiate the actual
fit to the baseline blocks selected in the previous step.  This creates a new plot
with the baseline drift removed as shown below.

\noindent\sphinxincludegraphics{{base_corrected}.png}

\sphinxAtStartPar
The default baseline fit is a 1st order polynomial, however, this can be adjusted accordingly
using the “Fit order” toggle. There are also options to filter the data (smoothing/removing RFI)
if needed, with a smoothing window provided to cater for that. After the baseline is corrected
the equation of a straight line as well ast the points used for the fit are displayed
as well. When fitting the peak, the peak selection works the same way as the baseline
selection with the exception
that the “Fit Location” toggle is set to “Peak”. Depending on where you want to fit
around the peak, you need to select peak fitting points as shown below.

\noindent\sphinxincludegraphics{{peak_select}.png}

\sphinxAtStartPar
Once you are happy with the selection points, you press “Fit data” again so the program
fits the peak of the corrected data. This will create a new fit overlayed on the plot
with information on the peak fit and its error. It is this value that we use to calculate
the PSS or the source flux depending on the target object under investigation.

\noindent\sphinxincludegraphics{{peak_fit}.png}

\sphinxAtStartPar
When you are happy with the fit, you need to save it using the “Save” button. This saves the
changes made to the current session, in this example, we are saving changes made to the
LCP ON SCAN drift scan. The minute you press save, any previous data you had previously
processed on this observation for this scan will be replaced by the changes you made.
This will also update the previous image you had. If you want to make changes to other
sessions as well you can use the “Current plot selection” toggle to change to new
drift scan session and load a different set of data. It is imperative to remember that
for any changes to be save you need to click on the “Save” button for each session or
each different drift scan. To view the values that will be modified
on the database, click on “View fit”. At this point the changes are still local to your
session, so to make the changes permanent on the database you need to click on “Save to
DB”. This updates the database and makes your changes permanent. A popup
will appear and ask if you are sure you want to continue modifying the values
in the database, If you are, you click yes and end your session. If you want
for example to revert to the previously automated fit that you accidentally
modified using the GUI, you need to go into the database and delete
the observation from there, then run the code again in automated mode to re\sphinxhyphen{}process
the automated fit.

\noindent\sphinxincludegraphics{{db_save}.png}


\section{Edit Time series}
\label{\detokenize{extras/gui:edit-time-series}}
\sphinxAtStartPar
Currently this feature is not fully operational. The plan is to have this gui
allow you to modify timeseries data manually. Right now, all you can do is view
the timeseries data and view the drift scan plots responsible for the points
on the time series data.

\sphinxAtStartPar
Opening a database in “Edit Time series” lets you plot any timeseries on the data
provided. For this example I’m plotting observed date vs PSS.

\noindent\sphinxincludegraphics{{db}.png}

\sphinxAtStartPar
To view the drift scan plots that provided a specific point on the timeseries you
click on the point in the timeseries and a popup html will show up with all the
processed scans for that specific point.

\noindent\sphinxincludegraphics{{insert}.png}

\begin{sphinxadmonition}{note}{Note:}
\sphinxAtStartPar
In future releases the GUI will be adapted to also handle timeseries analysis.
\end{sphinxadmonition}


\section{View plots}
\label{\detokenize{extras/gui:view-plots}}
\sphinxAtStartPar
View plots lets you view the thumbnails of all the plots made by the code for each
observed object.

\noindent\sphinxincludegraphics{{plot_views}.png}

\sphinxAtStartPar
Selecting the object name from a dropdown list provies a tile view
of all the plots made for that current object. Clickin on the tiles
provides a zoomed in image of the thumbnail.

\noindent\sphinxincludegraphics{{plots}.png}

\sphinxAtStartPar
Currently this does not work well when there is a lot of data
because the page takes a long time to load. In future work this will be modified
for effecient page loading.

\sphinxstepscope


\section{Radio frequency interference (RFI)}
\label{\detokenize{extras/rfi:radio-frequency-interference-rfi}}\label{\detokenize{extras/rfi::doc}}
\sphinxAtStartPar
\sphinxstylestrong{RFI} is a problem for many astronomers.

\sphinxstepscope


\section{Tutorials}
\label{\detokenize{extras/tuts/tutorials:tutorials}}\label{\detokenize{extras/tuts/tutorials::doc}}
\sphinxAtStartPar
The following tutorial will show you how to run a basic data
reduction for a Hydra A 13cm / 2 GHz observation.


\subsection{Data reduction of 2 GHz data.}
\label{\detokenize{extras/tuts/tutorials:data-reduction-of-2-ghz-data}}
\sphinxAtStartPar
First we need to read in data from the file we want to process

\begin{sphinxVerbatim}[commandchars=\\\{\}]
\PYGZdl{}\PYG{+w}{ }python\PYG{+w}{ }dran.py\PYG{+w}{ }\PYGZhy{}f\PYG{+w}{ }test\PYGZus{}data/HydraA\PYGZus{}13NB/2011d285\PYGZus{}04h55m29s\PYGZus{}Cont\PYGZus{}mike\PYGZus{}HYDRA\PYGZus{}A.fits
\end{sphinxVerbatim}

\sphinxAtStartPar
This will produce the following output

\begin{sphinxVerbatim}[commandchars=\\\{\},numbers=left,firstnumber=1,stepnumber=1]
************************************************************
\PYG{c+c1}{\PYGZsh{} PROCESSING SOURCE:}
************************************************************
\PYG{c+c1}{\PYGZsh{} File name:  2011d285\PYGZus{}04h55m29s\PYGZus{}Cont\PYGZus{}mike\PYGZus{}HYDRA\PYGZus{}A.fits}
\PYG{c+c1}{\PYGZsh{} Object:  HYDRA A}
\PYG{c+c1}{\PYGZsh{} Object type:  CAL}
\PYG{c+c1}{\PYGZsh{} Central Freq:  2280.0}
\PYG{c+c1}{\PYGZsh{} Observed :  2011\PYGZhy{}10\PYGZhy{}12}
************************************************************

\PYGZhy{}\PYG{+w}{ }Processing:\PYG{+w}{ }ON\PYGZus{}LCP


\PYG{c+c1}{\PYGZsh{} No sidelobes detected}


*\PYG{+w}{ }Center\PYG{+w}{ }of\PYG{+w}{ }baseline\PYG{+w}{ }blocks\PYG{+w}{ }on\PYG{+w}{ }left\PYG{+w}{ }and\PYG{+w}{ }right\PYG{+w}{ }of\PYG{+w}{ }peak:
min\PYG{+w}{ }pos\PYG{+w}{ }left:\PYG{+w}{ }\PYG{o}{[}\PYGZhy{}0.4041615\PYG{o}{]}\PYG{+w}{ }@\PYG{+w}{ }loc/s\PYG{+w}{ }\PYG{o}{[}\PYG{l+m}{160}\PYG{o}{]}
min\PYG{+w}{ }pos\PYG{+w}{ }right:\PYG{+w}{ }\PYG{o}{[}\PYG{l+m}{0}.391495\PYG{o}{]}\PYG{+w}{ }@\PYG{+w}{ }loc/s\PYG{+w}{ }\PYG{o}{[}\PYG{l+m}{2497}\PYG{o}{]}
scan\PYG{+w}{ }len:\PYG{+w}{ }\PYG{l+m}{2699}

\PYG{c+c1}{\PYGZsh{} Fit the baseline}
************************************************************

\PYG{c+c1}{\PYGZsh{} Fit = 0.073x + (\PYGZhy{}0.034), rms error = 0.0292}

\PYG{c+c1}{\PYGZsh{} Fit the peak}
************************************************************

\PYG{c+c1}{\PYGZsh{} Peak = 2.464 +\PYGZhy{} 0.030 [K]}

\PYG{c+c1}{\PYGZsh{} S/N: 86.77}

\PYGZhy{}\PYG{+w}{ }Processing:\PYG{+w}{ }ON\PYGZus{}RCP


\PYG{c+c1}{\PYGZsh{} No sidelobes detected}


*\PYG{+w}{ }Center\PYG{+w}{ }of\PYG{+w}{ }baseline\PYG{+w}{ }blocks\PYG{+w}{ }on\PYG{+w}{ }left\PYG{+w}{ }and\PYG{+w}{ }right\PYG{+w}{ }of\PYG{+w}{ }peak:
min\PYG{+w}{ }pos\PYG{+w}{ }left:\PYG{+w}{ }\PYG{o}{[}\PYGZhy{}0.40816145\PYG{o}{]}\PYG{+w}{ }@\PYG{+w}{ }loc/s\PYG{+w}{ }\PYG{o}{[}\PYG{l+m}{153}\PYG{o}{]}
min\PYG{+w}{ }pos\PYG{+w}{ }right:\PYG{+w}{ }\PYG{o}{[}\PYG{l+m}{0}.39016168\PYG{o}{]}\PYG{+w}{ }@\PYG{+w}{ }loc/s\PYG{+w}{ }\PYG{o}{[}\PYG{l+m}{2525}\PYG{o}{]}
scan\PYG{+w}{ }len:\PYG{+w}{ }\PYG{l+m}{2731}

\PYG{c+c1}{\PYGZsh{} Fit the baseline}
************************************************************

\PYG{c+c1}{\PYGZsh{} Fit = \PYGZhy{}0.000632x + (\PYGZhy{}0.0406), rms error = 0.0254}

\PYG{c+c1}{\PYGZsh{} Fit the peak}
************************************************************

\PYG{c+c1}{\PYGZsh{} Peak = 2.493 +\PYGZhy{} 0.025 [K]}

\PYG{c+c1}{\PYGZsh{} S/N: 100.69}
\end{sphinxVerbatim}

\sphinxAtStartPar
We are now going to breakdown the results returned

\sphinxAtStartPar
Lines 1 \sphinxhyphen{} 9 give us basic details on the source under observation.
This includes the name of the file being processed, the object
being observed, the type of object it is (CAL = Calibrator or
TAR = Target), the observing frequency, as well as the date the
source was observed.

\sphinxAtStartPar
Line 11 tells you the drift scan currently being processed, in this
case its the LCP On scan drift scan.

\sphinxAtStartPar
Line 14 is a debugging output that lets you know if any large sidelobes
were detected, these are sidelobes which are larger than half the peak
maximum.

\sphinxAtStartPar
Once the data is loaded and prepped, the program begins processing
the data. First it tries to correct or remove any drift in the data
that may exists. Using a gradient descent type algorithm, the program
fits a spline through the data and detects the location of the lowest
minimum locations on either sides of the center of the drift scan.

\sphinxAtStartPar
Line 17 \sphinxhyphen{} 20 give us information on the positions selected as the
local minimum points. An also gives the length of the scan.

\sphinxAtStartPar
4\% of the scan length is then used as the number of points required to
get enough data around the local minimum points in order to fit a
polynomila therough the data.

\sphinxAtStartPar
The equation of the line that is used to correct the drift in the data
is then displayed in line 25.

\sphinxstepscope


\section{Commands}
\label{\detokenize{extras/commands:commands}}\label{\detokenize{extras/commands::doc}}
\sphinxAtStartPar
The following commands are required when optiong to
run \sphinxhref{cli.rst}{CLI} analysis.

\sphinxstepscope


\section{Changelog}
\label{\detokenize{extras/changelog:changelog}}\label{\detokenize{extras/changelog::doc}}

\subsection{0.0.1 (2023\sphinxhyphen{}08\sphinxhyphen{}16)}
\label{\detokenize{extras/changelog:id1}}\begin{itemize}
\item {} 
\sphinxAtStartPar
First release on READ THE DOCS. Still a work in progress

\end{itemize}



\renewcommand{\indexname}{Index}
\printindex
\end{document}